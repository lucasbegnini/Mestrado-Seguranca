\subsection{Defini\c{c}\~oes de seguran\c{c}a}
\label{subsection:definicoes_seguranca}
Como \cite{o2003comparing} fala em seu artigo as defini\c{c}\~oes relativas aos sistemas e m\'etodo em seguran\c{c}a s\~ao definidas como forte ou fraca. Isso acontece pois quando usamos termos relativos, segundo o mesmo, somos claros na mensagem que queremos passar a respeito da informa\c{c}\~ao. Por exemplo, quando dizemos que uma porta com uma trava \'e mais segura, mais forte, que uma porta que n\~ao possui trava alguma, relativamos a seguran\c{c}a da trava tornando claro a mensagem que por mais que seja mais segura n\~ao \'e imposs\'ivel de ser quebrada.
\newline
Por mais que a relativiza\c{c}\~ao muitas vezes seja vista com maus olhos no campo da inform\'atica a garantia de um sistema inviol\'avel \'e quase imposs\'ivel e quando poss\'ivel altamente custoso. 
\newline
Como podemos ver no livro do \cite{stuttard2011web} h\'a diversas maneiras de se burlar um sistema web por exemplo, mesmo sendo um sistema altamente visado e onde pessoas se preocupam o tempo todo em proteg\^e-lo. Mas um ponto comum de quase todo ataque \'e o qu\^e conhecemos como engenharia social. A entrada sempre acontece por meio de pessoas, ou fragilidade dessas pessoas.
\newline
\'E muito dif\'icil mensurar seguran\c{c}a em termos absolutos. Uma forma de mensurar \'e atrav\'es da for\c{c}a e da fraqueza de um sistema. Um sistema forte \'e aquele que o custo de atacar \'e muito maior que o ganho o atacando. Ou seja, o trabalho para consegui a informa\c{c}\~ao vai ser t\~ao grande que n\~ao compensar\'a. Um exemplo \'e em \cite{biryukov2010key} a alta complexidade e tempo para se quebrar uma chave AES-256.
\newline
Paralelamente \`a isso temos a fraqueza de um sistema. Essa \'e o oposto da for\c{c}a. Ou seja, o custo de ataque \'e menor do o ganho obtido com os resultados. E de que custo podemos entender n\~ao s\'o dinheiro como tamb\'em tempo, potencial para puni\c{c}\~ao criminal e etc.