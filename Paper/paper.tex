\documentclass[12pt]{article}

\usepackage{sbc-template}

\usepackage{graphicx,url}
\usepackage[brazil]{babel}
%\usepackage[latin1]{inputenc}  
\usepackage[utf8]{inputenc}
\usepackage{import}
\usepackage{float}
% UTF-8 encoding is recommended by ShareLaTex

     
\sloppy

\title{A survey into CDNs Security}

\author{Lucas Begnini Costa\inst{1}, Carlos A. Maziero\inst{1}}


\address{LARSIS - Departamento de Informatica -- Universidade Federal do Paran\'a
  (UFPR)\\
 Curitiba -- PR -- Brazil
 \email{lucasbegnini@gmail.com,}
}

\begin{document} 

\maketitle

\begin{abstract}
  This meta-paper describes the style to be used in articles and short papers for SBC conferences. For papers in English, you should add just an abstract
  while for the papers in Portuguese, we also ask for an abstract in
  Portuguese (``resumo''). In both cases, abstracts should not have more than
  10 lines and must be in the first page of the paper.
\end{abstract}
     
\begin{resumo} 
  Este meta-artigo descreve o estilo a ser usado na confecção de artigos e
  resumos de artigos para publicação nos anais das conferências organizadas
  pela SBC. É solicitada a escrita de resumo e abstract apenas para os artigos
  escritos em português. Artigos em inglês deverão apresentar apenas abstract.
  Nos dois casos, o autor deve tomar cuidado para que o resumo (e o abstract)
  não ultrapassem 10 linhas cada, sendo que ambos devem estar na primeira
  página do artigo.
\end{resumo}


\section{Introdu\c{c}\~ao}


Vivemos em um mundo rodeado de tecnologia, onde a cada dia somos surpreendidos com uma coisa totalmente inovadora, disruptiva. Uma era onde a internet foi respons\'avel por atravessar mares, superar dist\^ancias e at\'e mesmo idiomas. Hoje se pode comunicar em tempo real com pessoas que est\~ao em lados completamente opostos ao seu. 
\newline
Hoje \'e poss\'ivel que empresas estrangeiras muito distantes fisicamente, como China, India, EUA e etc, forne\c{c}am servi\c{c}os para regi\~oes mais remotas do mundo. Isso inclui servi{c}os de multim\'idia como armazenamento de fotos, de v\'ideos e at\'e conte\'udos de consumo instant\^aneo.
\newline
Esse encurtamento de dist\^ancia pode parecer simples, mas vem de um sistema complexo que visa fornecer ao usu\'ario final uma experi\^encia agrad\'avel com conte\'udos entregues de maneira satifat\'oria mas sem, necessariamente, replica-lo por todo o globo. O que nos leva a dizer que uma CDN(\textit{Content Delivery Network} ) \'e uma rede de distribui\c{c}\~ao de conte\'udo que tem como objetivo fornecer ao usu\'ario de aplica\c{c}\~oes globais uma experi\^encia satisfat\'oria na utiliza\c{c}\~ao de servi\c{c}os, principalmente sob demanda.
\begin{figure}[H]
\includegraphics[height=7cm]{Figuras/contextualizacao.png} 
\label{figura:contextualizacao} 
\end{figure}
Temos v\'arios servi\c{c}os que se utiliza no dia a dia onde essa no\c{c}\~ao de CDN \'e completamente abstrata ao usu\'ario final. Como servi\c{c}os de Video-On-Demand de empresas de TV, spotify, Amazon Prime Video e at\'e Netflix, como \'e mostrado no artigo do \cite{adhikari2012unreeling}.
\newline
Existem hoje diversas empresas que fornecem esse servi\c{c}o ao redor do globo. Como:
\begin{itemize}
\item Akamai;
\item Limelight;
\item Level 3;
\item e etc.
\end{itemize}

Essas tr\^es redes s\~ao hoje, as principais fornecedoras de servi\c{c}o de CDN da Netflix. Cada um com um caracter\'istica e voltado pra um p\'ublico.

\subsection{AKAMAI}
\paragraph{Origem}- Massachusetts Institute of Technology (MIT), em 1995, com Tim Berners-Lee.
\paragraph{Ponto de atua\c{c}\~ao}-
\begin{figure}[H]
\caption{Rede de distribui\c{c}\~ao Akamai}
\includegraphics[width=15cm]{Figuras/akamai_map.png} 
\label{figura:akamai_map}
\end{figure}
\paragraph{Principais clientes}- Adobe, Airbnb, American Idol, Audi, Autodesk, EMC2, e muitas outras.
\subsection{Limelight}
\paragraph{Origem}- 
\paragraph{Ponto de atua\c{c}\~ao}-
\begin{figure}[H]
\caption{Rede de distribui\c{c}\~ao Limelight}
\includegraphics[width=15cm]{Figuras/limelight_map.png} 
\label{figura:limelight_map}
\end{figure}
\paragraph{Principais clientes}-
\subsection{Level 3}
\paragraph{Origem}- 
\paragraph{Ponto de atua\c{c}\~ao}-
\begin{figure}[H]
\caption{Rede de distribui\c{c}\~ao Level 3}
\includegraphics[width=15cm]{Figuras/level3_map.png} 
\label{figura:level3_map}
\end{figure}
\paragraph{Principais clientes}-

\section{Composi\c{c}\~ao de uma CDN} \label{sec:composicao}

Para entendermos melhor uma CDN precisamos primeiro destrich\'a-la em v\'arios pequenos peda\c{c}os para assim compreende-la em uma maneira global.
\newline
Uma CDN apesar de abstratamente ser vista como um mecanismo \'unico, pode ser vista tamb\'em como a soma de v\'arios tipos de elementos, v\'arias caracter\'isticas que somadas e configuradas formar\'a um mecanismo \'unico e transparente aos usu\'arios. Caracter\'isticas essas que s\~ao organiza\c{c}\~ao, tipos de servidores, protocolos e tipos de conte\'udo.

\paragraph{Organiza\c{c}\~ao}- Quanto a organiza\c{c}\~ao uma CDN pode ser uma rede unicamente CDN ou uma rede \textit{overlay}, que nada mais \'e que uma rede onde ela tenta abstrai as camadas de redes j\'a existentes(como transporte, redes entre outras) e transforma-l\'a em uma rede puramente CDN.

\paragraph{Tipos de conte\'udos}- Os tipos de conte\'udo que ir\~ao ser transportados dentro da rede s\~ao fundamentais para definir diversos aspectos de configura\c{c}\~oes que ser\~ao utilizadas dentro da rede. Como por exemplo, a forma de Cache que ser\~ao feitas os arquivos ou at\'e mesmo a forma como v\~ao ser distribu\'idos esses mesmos conte\'udos, se ser\~ao distribu\'idos em conjunto ou em partes, como o caso de uma pagina HTML que possui um video para cada regi\~ao do  mundo. 
\newline
Todos os outros pontos levam em conta primeiro o tipo de conte\'udo para definir quais ser\~ao suas escolhas.
\newline
Os demais itens ser\~ao tratados nos pr\'oximos pontos. Tipos de servidores em \ref{section:tipos_de_servidores} e protocolos de intera\c{c}\~oes em \ref{section:protocolos_interacoes}

\subimport{composicao/}{tiposServidores}

\subimport{composicao/}{protocolosInteracoes}

\subimport{composicao/}{selecaoEntrega}


\section{Seguran\c{c}a de uma CDN} \label{sec:seguranca}

\subimport{seguranca/}{autenticacaousuario}
\subimport{seguranca/}{autenticacaoconteudo}
\subimport{seguranca/}{modelosdeataques}

\newpage
\section{References}

\bibliographystyle{sbc}
\bibliography{sbc-template}

\end{document}
